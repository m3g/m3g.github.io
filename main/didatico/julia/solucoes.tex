\documentclass[10pt,a4paper,ssfamily]{exam}
\usepackage[numbers,sort&compress]{natbib}
\usepackage[utf8]{inputenc}
\renewcommand{\baselinestretch}{1.15}
\usepackage{epsfig}
\usepackage{graphicx}
\usepackage{makeidx}
\usepackage{multicol}    
\usepackage{multirow}
\usepackage{amssymb}
\usepackage{enumitem}
\usepackage{xcolor}
\usepackage{lmodern}
\usepackage{scrextend}
\usepackage{xcolor}
\usepackage[hidelinks]{hyperref}
\newcommand{\listfont}{\bf\fontencoding{OT1}\sffamily\large}
\newcommand{\titlesfont}{\fontencoding{OT1}\sffamily\large}
\renewcommand{\baselinestretch}{1.24}
\newcommand{\oC}{$^\circ$C}
\newcommand{\sen}{\textrm{sen }}
\newcommand{\Mg}{Mg$^{2+}$}
\newcommand{\sol}[1]{\href{http://leandro.iqm.unicamp.br/shtml/didatico/qf935/atividades/#1.f90}{Solução}}
\renewcommand{\t}{\textrm}

\newsavebox{\fmbox}
\newenvironment{ex}
{\begin{center}\vspace{0.3cm}\begin{lrbox}{\fmbox}\begin{minipage}{12cm}{\bf Atividade}
\begin{enumerate}\setcounter{enumi}{\exnumber}}
{\global\chardef\exnumber=\value{enumi}\end{enumerate}\end{minipage}\end{lrbox}\fbox{\usebox{\fmbox}}\vspace{0.3cm}\end{center}}

% Code blocks definitions

\usepackage{listings}

\renewcommand{\t}{\textrm}
\lstnewenvironment{code}{\lstset{language=[90]Fortran,
  basicstyle=\ttfamily,
  keywordstyle=\color{blue},
  commentstyle=\color{gray},
  identifierstyle=\color[RGB]{0,102,0}\textbf,
  xleftmargin=1.5cm,
  showstringspaces=false,
  morecomment=[l]{!\ }% Comment only with space after !
}}{}

\lstdefinestyle{code}{
  language=[90]Fortran, 
  basicstyle=\ttfamily,
  keywordstyle=\color{blue},
  commentstyle=\color{gray},
  identifierstyle=\color[RGB]{0,102,0}\textbf,
  xleftmargin=1.5cm,
  showstringspaces=false,
  morecomment=[l]{!\ }% Comment only with space after !
}
\newcommand{\codefile}[1]{\lstinputlisting[style=code]{#1}
            \begin{center}
            \href{http://leandro.iqm.unicamp.br/leandro/shtml/didatico/simulacoes/#1}
                 {\textcolor{blue}{[Clique para baixar o código]}}
            \end{center}}

\begin{document}


% Start exercise counter
\global\chardef\exnumber=0
\sffamily

\pagestyle{empty}

\begin{center}
{\bf\Large Fundamentos Computacionais de Simulações em Química}

Leandro Martínez\\
leandro@iqm.unicamp.br \\
\end{center}

\begin{center}
{\bf Soluções de atividades selecionadas}
\end{center}

\subsubsection*{Atividade 6}
\begin{code}
program at6
  integer :: i, j
  i = 1
  do while( i > 0 ) 
    j = i
    i = i + 1
  end do
  write(*,*) ' Greatest integer = ', j
end program at6
\end{code}

Comentários: O programa vai naturalmente ser mais rápido se, em lugar de
começar com {\tt i=1}, começarmos com um valor próximo do maior inteiro.
Aqui são definidas duas variáveis inteiras, {\tt i} e {\tt j}, sendo que
{\tt j} assume o valor de {\tt i} {\it antes} de que este seja
modificado. Assim, quando o {\tt i} ultrapassar o maior inteiro
representável, e se tornar negativo, o {\it loop} termina e {\tt j}
preserva o valor anterior à última modificação.

\vspace{1cm}
\subsubsection*{Atividade 14}

O loop do programa que propaga as concentrações usando a discretização
das equações diferencias deve ser algo como:
\begin{code}
do i = 1, nsteps
  CA = CA - k1*CA*dt + km1*CB*dt ! Concentration of A
  CB = CB + k1*CA*dt - km1*CB*dt ! Concentration of B
  error = ( CA + CB ) - ( CA0 + CB0 ) ! Testing balance of mass
  time = time + dt
  write(*,*) time, CA, CB, error
end do
\end{code}
Dentro deste loop, deve ser adicionado um teste sobre o erro, que deve
sair do loop e escrever uma mensagem de erro caso a diferença da soma
das concentrações com relação às concentrações iniciais seja muito
grande:
\begin{code}
...
write(*,*) time, CA, CB, error
if ( error > 1.d-3 ) then
  write(*,*) ' ERROR: Balance of mass failed. error = ', error
  exit
end if
...
\end{code}
O {\tt exit} acima vai terminar o loop, e o programa termina depois
disto. Poderíamos, também, parar o programa diretamente, usando {\tt
stop}. No exemplo, escolheu-se um erro de $10^{-3}$ como erro máximo
tolerável. Naturalmente, o erro máximo tolerável depende do problema.

\vspace{1cm}
\subsection*{Atividade 18}

No código abaixo, foram introduzidas duas variávies, {\tt i} e {\tt
ntrial}, e o loop foi modificado para dar no máximo {\tt ntrial} voltas.
Além disso, removemos a parte do código que parava o programa no caso de
aumento do valor da função. Compare com o programa {\tt min1}.

\codefile{./codes/atividade18.f90}

\vspace{1cm}
\subsection*{Atividade 21}

\codefile{./codes/min2.f90}

\vspace{1cm}
\subsection*{Atividade 29}

Nesta atividade as rotinas de cálculo da função e do gradiente são
separadas do programa principal. Há alguns detalhes importantes: Os
vetores são declarados, aqui, com dimensões {\it fixas} no programa
principal, por exemplo, {\tt x(1000)}. Nas subrotinas, declaramos os
vetores usando, por exemplo, {\tt x(n)}, sendo {\tt n} um parâmetro de
entrada da subrotina. Esta declaração, dentro da subrotina é, na
verdade, apenas um lembrete de quantos elementos do vetor vão se usados.
O que o programa passa para a subrotina é apenas o {\it endereço na
memória} do vetor. Isto é, o programa principal diz à subrotina:
``trabalhe com este vetor, está neste lugar na memória''. O tamanho
efetivo do vetor é aquele declarado no programa principal, e a
declaração na subrotina é, na verdade, redundante. De fato, a mesma
declaração {\tt x(n)} poderia ser feita usando {\tt x(1)}, ou mesmo {\tt
x(*)}. Note, também, que como critério de parada usamos o quadrado da
norma do gradiente.     

\codefile{./codes/ativ29.f90}

\vspace{1cm}
\subsection*{Atividade 34}
\codefile{./codes/randomsearch2.f90}




\end{document}

